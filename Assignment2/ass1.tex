
%=======================   Default Templete   ==================
\documentclass[a4paper]{article}


% file with some default definations
%%%%%%%%%%%%%%%%%%%%%%%%%%%%%%%%%%%%%%%%%
% Lachaise Assignment
% Structure Specification File
% Version 1.0 (26/6/2018)
%
% This template originates from:
% http://www.LaTeXTemplates.com
%
% Authors:
% Marion Lachaise & François Févotte
% Vel (vel@LaTeXTemplates.com)
%
% License:
% CC BY-NC-SA 3.0 (http://creativecommons.org/licenses/by-nc-sa/3.0/)
% 
%%%%%%%%%%%%%%%%%%%%%%%%%%%%%%%%%%%%%%%%%

%----------------------------------------------------------------------------------------
%	PACKAGES AND OTHER DOCUMENT CONFIGURATIONS
%----------------------------------------------------------------------------------------

\usepackage{amsmath,amsfonts,stmaryrd,amssymb} % Math packages

\usepackage{enumerate} % Custom item numbers for enumerations

\usepackage[ruled]{algorithm2e} % Algorithms

\usepackage[framemethod=tikz]{mdframed} % Allows defining custom boxed/framed environments

\usepackage{listings} % File listings, with syntax highlighting
\lstset{
	basicstyle=\ttfamily, % Typeset listings in monospace font
}

%----------------------------------------------------------------------------------------
%	DOCUMENT MARGINS
%----------------------------------------------------------------------------------------

\usepackage{geometry} % Required for adjusting page dimensions and margins

\geometry{
	paper=a4paper, % Paper size, change to letterpaper for US letter size
	top=2.5cm, % Top margin
	bottom=3cm, % Bottom margin
	left=2.5cm, % Left margin
	right=2.5cm, % Right margin
	headheight=14pt, % Header height
	footskip=1.5cm, % Space from the bottom margin to the baseline of the footer
	headsep=1.2cm, % Space from the top margin to the baseline of the header
	%showframe, % Uncomment to show how the type block is set on the page
}

%----------------------------------------------------------------------------------------
%	FONTS
%----------------------------------------------------------------------------------------

\usepackage[utf8]{inputenc} % Required for inputting international characters
\usepackage[T1]{fontenc} % Output font encoding for international characters

\usepackage{XCharter} % Use the XCharter fonts

%----------------------------------------------------------------------------------------
%	COMMAND LINE ENVIRONMENT
%----------------------------------------------------------------------------------------

% Usage:
% \begin{commandline}
%	\begin{verbatim}
%		$ ls
%		
%		Applications	Desktop	...
%	\end{verbatim}
% \end{commandline}

\mdfdefinestyle{commandline}{
	leftmargin=10pt,
	rightmargin=10pt,
	innerleftmargin=15pt,
	middlelinecolor=black!50!white,
	middlelinewidth=2pt,
	frametitlerule=false,
	backgroundcolor=black!5!white,
	frametitle={Command Line},
	frametitlefont={\normalfont\sffamily\color{white}\hspace{-1em}},
	frametitlebackgroundcolor=black!50!white,
	nobreak,
}

% Define a custom environment for command-line snapshots
\newenvironment{commandline}{
	\medskip
	\begin{mdframed}[style=commandline]
}{
	\end{mdframed}
	\medskip
}

%----------------------------------------------------------------------------------------
%	FILE CONTENTS ENVIRONMENT
%----------------------------------------------------------------------------------------

% Usage:
% \begin{file}[optional filename, defaults to "File"]
%	File contents, for example, with a listings environment
% \end{file}

\mdfdefinestyle{file}{
	innertopmargin=1.6\baselineskip,
	innerbottommargin=0.8\baselineskip,
	topline=false, bottomline=false,
	leftline=false, rightline=false,
	leftmargin=2cm,
	rightmargin=2cm,
	singleextra={%
		\draw[fill=black!10!white](P)++(0,-1.2em)rectangle(P-|O);
		\node[anchor=north west]
		at(P-|O){\ttfamily\mdfilename};
		%
		\def\l{3em}
		\draw(O-|P)++(-\l,0)--++(\l,\l)--(P)--(P-|O)--(O)--cycle;
		\draw(O-|P)++(-\l,0)--++(0,\l)--++(\l,0);
	},
	nobreak,
}

% Define a custom environment for file contents
\newenvironment{file}[1][File]{ % Set the default filename to "File"
	\medskip
	\newcommand{\mdfilename}{#1}
	\begin{mdframed}[style=file]
}{
	\end{mdframed}
	\medskip
}

%----------------------------------------------------------------------------------------
%	NUMBERED QUESTIONS ENVIRONMENT
%----------------------------------------------------------------------------------------

% Usage:
% \begin{question}[optional title]
%	Question contents
% \end{question}

\mdfdefinestyle{question}{
	innertopmargin=1.2\baselineskip,
	innerbottommargin=0.8\baselineskip,
	roundcorner=5pt,
	nobreak,
	singleextra={%
		\draw(P-|O)node[xshift=1em,anchor=west,fill=white,draw,rounded corners=5pt]{%
		Question \theQuestion\questionTitle};
	},
}

\newcounter{Question} % Stores the current question number that gets iterated with each new question

% Define a custom environment for numbered questions
\newenvironment{question}[1][\unskip]{
	\bigskip
	\stepcounter{Question}
	\newcommand{\questionTitle}{~#1}
	\begin{mdframed}[style=question]
}{
	\end{mdframed}
	\medskip
}

%----------------------------------------------------------------------------------------
%	WARNING TEXT ENVIRONMENT
%----------------------------------------------------------------------------------------

% Usage:
% \begin{warn}[optional title, defaults to "Warning:"]
%	Contents
% \end{warn}

\mdfdefinestyle{warning}{
	topline=false, bottomline=false,
	leftline=false, rightline=false,
	nobreak,
	singleextra={%
		\draw(P-|O)++(-0.5em,0)node(tmp1){};
		\draw(P-|O)++(0.5em,0)node(tmp2){};
		\fill[black,rotate around={45:(P-|O)}](tmp1)rectangle(tmp2);
		\node at(P-|O){\color{white}\scriptsize\bf !};
		\draw[very thick](P-|O)++(0,-1em)--(O);%--(O-|P);
	}
}

% Define a custom environment for warning text
\newenvironment{warn}[1][Warning:]{ % Set the default warning to "Warning:"
	\medskip
	\begin{mdframed}[style=warning]
		\noindent{\textbf{#1}}
}{
	\end{mdframed}
}

%----------------------------------------------------------------------------------------
%	INFORMATION ENVIRONMENT
%----------------------------------------------------------------------------------------

% Usage:
% \begin{info}[optional title, defaults to "Info:"]
% 	contents
% 	\end{info}

\mdfdefinestyle{info}{%
	topline=false, bottomline=false,
	leftline=false, rightline=false,
	nobreak,
	singleextra={%
		\fill[black](P-|O)circle[radius=0.4em];
		\node at(P-|O){\color{white}\scriptsize\bf i};
		\draw[very thick](P-|O)++(0,-0.8em)--(O);%--(O-|P);
	}
}

% Define a custom environment for information
\newenvironment{info}[1][Info:]{ % Set the default title to "Info:"
	\medskip
	\begin{mdframed}[style=info]
		\noindent{\textbf{#1}}
}{
	\end{mdframed}
}

\usepackage{listings}
\lstset{language=Python, basicstyle=\normalsize\sffamily\linespread{0.8}, numbers=left, numberstyle=\small, stepnumber=1, numbersep=5pt}
\usepackage{fancyhdr}
\setlength{\parindent}{0pt}

\pagestyle{fancy}
\fancyhf{}
\lhead{\textbf{\NAME\ (\ANDREWID)}}
\chead{\textbf{Assignment \HWNUM}}
\rhead{\COURSE}

\renewcommand{\qedsymbol}{\rule{0.7em}{0.7em}}

%==================Header details======================
\newcommand\NAME{Raghukul, Vibhor}
\newcommand\ANDREWID{160538, 160778}
\newcommand\HWNUM{1}
\newcommand\COURSE{CS648}
%======================================================

% available formatted sections:
% - COMMAND LINE ENVIRONMENT: \begin{commandline} \end{commandline}
% - FILE CONTENTS ENVIRONMENT: \begin{file}[optional filename, defaults to "File"]
% - NUMBERED QUESTIONS ENVIRONMENT: \begin{question}[optional title]
% - WARNING TEXT ENVIRONMENT(can also be used for note): \begin{warn}[optional title, defaults to "Warning:"]
% - INFORMATION ENVIRONMENT(can be used to mention given details): \begin{info}[optional title, defaults to "Info:"]

%===============================================================
\begin{document}

% start of Q2
\begin{question}[]
\textbf{(1) Let $a_k$ denote the element of $S$ inserted during $k^{th}$ iteration in the above algorithm.
Let $\varepsilon_k$ be the event that $a_k$ is inserted in table $T_2$. Calculate $P(\varepsilon_k)$.\\
(2) Let $X_k$ denote the random variable defined as follows. If $a_k$ is stored in $T_1$, then
$X_k = 0$; otherwise, it is the number of elements among $\{a_1, . . . , a_{k-1}\}$ that collide with it in $T_2$.
Provide a suitable upper bound on $E[X_k | E_k]$.\\
(3)  Use (2) to show that we can design a perfect hashing (at most one element in any
location of $T_1$ and $T_2$) using $n$ which is asymptotically much smaller than $s^2$.}
\end{question}
\subsubsection*{(1)}
    Probability of event $\varepsilon_k$ would be the same as the probability of the event that $a_k$ collides with at least one of $a_1, a_2, ... , a_{k-1}$ under hash function $h_1$. Call this event to be $E$.
    
    Call the event that $a_k$ collides with $a_i$ to be $E_i$ where $1\leq i\leq (k-1)$
    Now, $$P(E) = P(\bigcup_{i=1}^{k-1} E_i)$$
    Since $h_1$ is selected randomly uniformly from universal hash family $H$ we get for any $a,b\ \epsilon\ U$,
    $$P(h_1(a) = h_1(b)) \leq \frac{c}{n}$$
    Thus,
    $$P(E_i) \leq \frac{c}{n}$$
    By union theorem,
    $$P(E) \leq P(\sum_{i=1}^{k-1} E_i)$$
    $$P(E) \leq \frac{(k-1)c}{n}$$
    also we have $P(\varepsilon_k) = P(E)$.
    Hence we get,
    $$P(\varepsilon_k) \leq \frac{(k-1)c}{n}$$

\subsubsection*{(3)}
\pagebreak



% start of Q3
\begin{question}[]
\textbf{Consider a collection $X_1, ..., X_n$ of $n$ independent geometrically distributed random variables with expected value $2$. Let $X = \sum_{i=1}^NX_i$ and $\delta > 0$\\
(1) Derive a bound on $P(X \geq (1 + \delta)(2n))$ by appyling the Chernoff bound to a sequence of $(1 + \delta)(2n)$ fair coin tosses.} \\
\textbf{(2) Directly derive a Chernoff like bound on $P(X \geq (1 + \delta)(2n))$ from scratch.} \\
\textbf{(3) Which bound is better?}

\end{question}
Let $Y$ be a geometrically distributed random variable then $P(Y=k) = (1-p)^{k-1}p$ and $E[Y]=\frac{1}{p}$.
\subsubsection*{(1)}
Since $E[X_i] = 2$ we find that each $X_i$ is geometrically distributed with parameter $p=\frac{1}{2}$.

As $p=\frac{1}{2}$ we can think of each $X_i$ as the number of coin tosses required to get the first head.

So their sum $X = \sum_{i=1}^nX_i$ can be thought of as the number of coin tosses required to get $n$ heads.

We have to find the probability that no less than $(1+\delta)(2n)$ coin tosses are required to get $n$ heads, this is same as the probability that less than $n$ heads appear in $(1+\delta)(2n)$ coin tosses which in turn is same as the probability that no less than $n(1+2\delta)$ tails appear in $(1+\delta)(2n)$ coin tosses

Let $Y_i$ be the random variable taking value $1$ when $i^{th}$ coin toss results in a tail and $0$ otherwise. Each $Y_i$ is a Bernoulli random variable with probability equal to $\frac{1}{2}$

Consider $Y = \sum_{i=1}^{(1+\delta)(2n)}Y_i$. By linearity of expectation, $E[Y] = (1+\delta)n$

Now, $$P(X \geq (1+\delta)(2n)) = P(Y \geq n(1+2\delta))$$
$$P(Y \geq n(1+2\delta)) = P\left(Y \geq \frac{(1+\delta)n(1+2\delta)}{1+\delta}\right)$$
$$P(Y \geq n(1+2\delta)) = P\left(Y \geq (1+\delta)n\left(1+\frac{\delta}{1+\delta}\right)\right)$$
$$P(Y \geq n(1+2\delta)) = P\left(Y \geq E[Y]\left(1+\frac{\delta}{1+\delta}\right)\right)$$
Applying chernoff bound on $Y$,
$$P(Y \geq n(1+2\delta)) \leq \frac{e^{n\delta}(1+\delta)^{n(1+2\delta)}}{(1+2\delta)^{n(1+2\delta)}}$$
$$P(Y \geq n(1+2\delta)) \leq \left(\frac{e^{\delta}(1+\delta)^{(1+2\delta)}}{(1+2\delta)^{(1+2\delta)}}\right)^n$$
Thus we get,
$$P(X \geq (1+\delta)(2n)) \leq \left(\frac{e^{\delta}(1+\delta)^{(1+2\delta)}}{(1+2\delta)^{(1+2\delta)}}\right)^n$$
\subsubsection*{(2)}
$$P(X \geq (1+\delta)(2n)) = P(e^{tX} \geq e^{t(1+\delta)(2n)})$$
By Markov's inequality,
$$P(e^{tX} \geq e^{t(1+\delta)(2n)}) \leq \frac{E[e^{tX}]}{e^{t(1+\delta)(2n)}}$$
Now,
$$E[e^{tX}] = E[e^{(tX_1 + ... + tX_n)}]$$
$$E[e^{tX}] = E[e^{tX_1}e^{tX_2}...e^{tX_n}]$$
$$E[e^{tX}] = E[\Pi_{i=1}^ne^{tX_i}]$$
We know that $X_i$s are independent so $e^{tX_i}$s are also independent amd hence we get,
$$E[e^{tX}] = \Pi_{i=1}^nE[e^{tX_i}]$$
and $$E[e^{tX_i}] = \sum_{j=1}^\infty e^{tj}P(X_i=j)$$
$$E[e^{tX_i}] = \sum_{j=1}^\infty e^{tj}(1-p)^{j-1}p$$
$$E[e^{tX_i}] = e^tp\sum_{j=1}^\infty {(e^t(1-p))}^{j-1}$$
We had $p=\frac{1}{2}$. Take $t < \ln 2$ so that $e^t(1-p) < 1$.

By using the formula of summation of a GP,
$$E[e^{tX_i}] = e^tp\frac{1}{1 - e^t(1-p)}$$
$$E[e^{tX}] = E[\Pi_{i=1}^ne^{tX_i}]$$
$$E[e^{tX}] = \left(\frac{e^tp}{1 - e^t(1-p)}\right)^n$$
Thus we get,
$$P(e^{tX} \geq e^{t(1+\delta)(2n)}) \leq \frac{1}{e^{t(1+\delta)(2n)}}\left(\frac{e^tp}{1 - e^t(1-p)}\right)^n$$
$$P(e^{tX} \geq e^{t(1+\delta)(2n)}) \leq \left(\frac{e^tp}{(1 - e^t(1-p))(e^{2t(1+\delta)})}\right)^n$$
This is true for every value of $t < \ln 2$ and hence differentiating it to find the minimum value - 
$$\frac{d}{dt}\left(\frac{e^tp}{(1 - e^t(1-p))(e^{2t(1+\delta)})}\right)^n = 0$$
$$\frac{d}{dt}\frac{e^tp}{(1 - e^t(1-p))(e^{2t(1+\delta)})} = 0$$
$$-(1-e^t(1-p))(1+2\delta)(e^{-t(1+2\delta)}) + (e^{-t(1+2\delta)})(e^t(1-p)) = 0$$
$$e^t(1-p) = (1+2\delta)(1-e^t(1-p))$$
$$e^t(1-p) = \frac{1+2\delta}{2+2\delta}$$
Put $p=\frac{1}{2}$,
$$e^t = \frac{1+2\delta}{1+\delta}$$
So we get after substituting this value of $e^t$,
$$P(e^{tX} \geq e^{t(1+\delta)(2n)}) \leq \left(\frac{(1+\delta)^{2+2\delta}}{(1+2\delta)^{1+2\delta}}\right)^n$$
So, $$P({X} \geq {(1+\delta)(2n)}) \leq \left(\frac{(1+\delta)^{2+2\delta}}{(1+2\delta)^{1+2\delta}}\right)^n$$

\subsubsection*{(3)}
Let the first and second bounds are $B1$ and $B2$ respectively,
$$\frac{B1}{B2} = \frac{e^\delta}{1+\delta}$$
$$\frac{B1}{B2} \geq 1$$
Thus, the second bound is better.
\end{document}
